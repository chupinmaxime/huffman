
% LTeX: language=en

% huffman : draw binary huffman trees with MetaPost/MetaObj
%
% Originally written by Maxime Chupin <notezik@gmail.com>,
% 2023.
%
% Distributed under the terms of the GNU free documentation licence:
%   http://www.gnu.org/licenses/fdl.html
% without any invariant section or cover text.

\documentclass[english]{ltxdoc}

\input{huffman-preamble}

\usepackage[english]{babel}

\makeindex[title=Command Index, columns=2]



%\lstset{moredelim=*[s][\color{red}\rmfamily\itshape]{<}{>}}
%\lstset{moredelim=*[s][\color{blue}\rmfamily\itshape]{<<}{>>}}

\begin{document}

\title{{Huffman} : drawing binary Huffman trees with \hologo{METAPOST} and \MO/}
\author{Maxime Chupin, \url{notezik@gmail.com}}
\date{\today}

%% === Page de garde ===================================================
\thispagestyle{empty}
\begin{tikzpicture}[remember picture, overlay]
  \node[below right, shift={(-4pt,4pt)}] at (current page.north west) {%
    \includegraphics{fond.pdf}%
  };
\end{tikzpicture}%

\noindent
{\Huge \texttt{huffman}}\par\bigskip
\noindent
{\Large  drawing binary Huffman trees \\[0.2cm]with \hologo{METAPOST} and \MO/}\\[1cm]
\parbox{0.6\textwidth}{
  \begin{mplibcode}
    input huffman

    string charList[];
numeric frequency[];
show_bits:=false;
beginfig(1);
charList[4]:="a"; frequency[4]:=4;
charList[5]:="b"; frequency[5]:=5;
charList[3]:="c"; frequency[3]:=6;
charList[1]:="d"; frequency[1]:=7;
charList[2]:="e"; frequency[2]:=10;
charList[6]:="f"; frequency[6]:=10;
charList[7]:="g"; frequency[7]:=18;
charList[8]:="h"; frequency[8]:=40;
newBinHuffman.myHuff(8)(charList,frequency) "treemode(T)", "vbsep(0.3cm)";
myHuff.c=origin;
drawObj(myHuff);
endfig;
  \end{mplibcode}
}\hfill
\parbox{0.5\textwidth}{\Large\raggedleft
  \textbf{Contributor}\\
  Maxime \textsc{Chupin}\\
  \url{notezik@gmail.com}
}
\vfill
\begin{center}
  Version 0.1, 2023, April, 21th \\
  \url{https://plmlab.math.cnrs.fr/mchupin/huffman}
\end{center}
%% == Page de garde ====================================================
\newpage

%\maketitle

\begin{abstract}
  This \hologo{METAPOST} package allows to draw binary Huffman trees from two
  arrays : a string one, and a value one. It is based on \MO/ package which
  provides many tools to build trees in general.
\end{abstract}


\begin{center}
  \url{https://plmlab.math.cnrs.fr/mchupin/huffman}
  \url{https://github.com/chupinmaxime/huffman}
\end{center}

\tableofcontents

\bigskip

\begin{tcolorbox}[ arc=0pt,outer arc=0pt,
  colback=darkred!3,
  colframe=darkred,
  breakable,
  boxsep=0pt,left=5pt,right=5pt,top=5pt,bottom=5pt, bottomtitle =
  3pt, toptitle=3pt,
  boxrule=0pt,bottomrule=0.5pt,toprule=0.5pt, toprule at break =
  0pt, bottomrule at break = 0pt,]
  \itshape
  This package is in beta version---do not hesitate to report bugs, as well as requests for improvement.
\end{tcolorbox}

\section{Installation}

\huffman is on \ctan{} and can also be installed via the package manager of your
distribution.

\begin{center}
  \url{https://www.ctan.org/pkg/huffman}
\end{center}


\subsection{With \TeX live under Linux or macOS}

To install \huffman with \TeX Live, you will have to create the directory
\lstinline+texmf+ directory in your \lstinline+home+. 

\begin{commandshell}
mkdir ~/texmf
\end{commandshell}

Then, you will have to place the \lstinline+huffman.mp+ file in the 
\begin{center}
  \lstinline+~/texmf/metapost/huffman/+
\end{center}


Once this is done, \huffman will be loaded with the classic \MP
input code
\begin{mpcode}
input huffman
\end{mpcode}

\subsection{With Mik\TeX{} and Windows}

These two systems are unknown to the author of \huffman, so we
refer you to the Mik\TeX documentation concerning the addition of local packages:
\begin{center}
  \url{http://docs.miktex.org/manual/localadditions.html}
\end{center}



\subsection{Dependencies}


\huffman depends, of course on \MP~\cite{ctan-metapost}, as well as the packages \package{metaobj}~\cite{ctan-metaobj}
and---if \huffman is not used with \hologo{LuaLaTeX} and the \package{luamplib}
package---the \package{latexmp} package.

\section{Main Command}

The package \huffman provides one principal command (which is a \MO/ like
constructor):

\commande|newBinHuffman.«name»(«sizeofarrays»)(«symbarray»,«valuearray»)|\smallskip\index{newBinHuffman@\lstinline+newBinHuffman+}


\begin{description}
  \item[\meta{name}:] is the name of the object;
  \item[\meta{sizeofarray}:] is the size (integer) of the arrays;
  \item[\meta{symbarray}:] is the array of \lstinline+string+ containing the
  symboles;
  \item[\meta{valuearray}:] is the array of \lstinline+numeric+ containing the
  values associated to the symboles.
\end{description}

The data arrays should begin at index 1.  

\begin{ExempleMP}
input huffman

beginfig(0);
string charList[];
numeric frequency[];
charList[1]:="a"; frequency[1]:=0.04;
charList[2]:="b"; frequency[2]:=0.05;
charList[3]:="c"; frequency[3]:=0.06;
charList[4]:="d"; frequency[4]:=0.07;
charList[5]:="e"; frequency[5]:=0.1;
charList[6]:="f"; frequency[6]:=0.1;
charList[7]:="g"; frequency[7]:=0.18;
charList[8]:="h"; frequency[8]:=0.4;

newBinHuffman.myHuff(8)(charList,frequency);
myHuff.c=origin;
drawObj(myHuff);
endfig;
\end{ExempleMP}

Beware, the symbols are composed in mathematical \TeX{} mode. 
\section{Package Options}

You can modify the size of the internal nodes of the tree with the following
command:

\commande|set_node_size(«dim»)|\smallskip\index{set_node_size@\lstinline+set_node_size+}
\begin{description}
  \item[\meta{dim}:] is the diameter of the circle with unity (default: 13pt).
\end{description} 

You can change the color for the symbol boxes with the following command:

\commande|set_leaf_color(«color»)|\smallskip\index{set_leaf_color@\lstinline+set_leaf_color+}
\begin{description}
  \item[\meta{color}:] is a \hologo{METAPOST} \lstinline+color+.
\end{description} 

You can hide the bit in the edges of the tree with the following boolean
(\lstinline+true+ by default):

\commande|show_bits|\smallskip\index{show_bits@\lstinline+show_bits+}

Similarly, you can set the following boolean to \lstinline+false+ to hide the
node values: 

\commande|show_node_values|\smallskip\index{show_node_values@\lstinline+show_node_values+}

Finally, you can set the following boolean to \lstinline+false+ to hide the
leaf values:

\commande|show_leaf_values|\smallskip\index{show_leaf_values@\lstinline+show_leaf_values+}

Here an example combining all these commands and variables.

\begin{ExempleMP}
input huffman

beginfig(0);
string charList[];
numeric frequency[];
charList[1]:="s_1"; frequency[1]:=2;
charList[2]:="s_2"; frequency[2]:=4;
charList[3]:="s_3"; frequency[3]:=2;
charList[4]:="s_4"; frequency[4]:=12;
charList[5]:="s_5"; frequency[5]:=8;

set_leaf_color(0.2[white,green]);
set_node_size(8pt);
show_bits:=false;
show_node_values:=false;
show_leaf_values:=false;
newBinHuffman.myHuff(5)(charList,frequency);
myHuff.c=origin;
drawObj(myHuff);
endfig;
\end{ExempleMP}
  


\section{\MO/ Tree Options}

Because the Huffman tree is build using \MO/ tree constructor, the \MO/ tree
options are available. All of them are not well suited for this application
mostly because the Huffman tree is build using elementary trees, to which the
options we give to the Huffman constructor is passed to all the subtrees.

We give in table~\ref{tab:options} the \MO/ options for the trees that could be
used for the Huffman constructor.

\begin{table}[ht]
\centering\small
\begin{tabular}{HHHp{5.5cm}}
  \toprule 
  Option & Type & Default & Description \\
  \midrule 
  treemode & string & "D" & direction in which the tree develops; there are
  four different possible values: \lstinline+"D"+ (default),
  \lstinline+"U"+, \lstinline+"L"+ and \lstinline+"R"+\\
  treeflip & boolean & false & if true, reverses the order of the subtrees\\
  treenodehsize & numeric & -1pt & if non-negative, all the nodes are assumed to
  have this width\\
  treenodevsize & numeric & -1pt & if non-negative, all the nodes are assumed to
  have this height\\
  dx & numeric & 0 & horizontal clearance around the tree\\
  dy & numeric & 0 & vertical clearance around the tree\\
  hsep & numeric & 1cm & for a horizontal tree, this is the separation be-
  tween the root and the subtrees\\
  vsep & numeric & 1cm & for a vertical tree, this is the separation be-
  tween the root and the subtrees\\
  hbsep & numeric & 1cm &for a vertical tree, this is the horizontal separation between subtrees;
  the subtrees are actually put in a \lstinline+HBox+ and the value of this
  option is passed to the \lstinline+HBox+ constructor\\
  vbsep & numeric & 1cm & for an horizontal tree, this is the vertical separation between subtrees;
  the subtrees are actually put in a \lstinline+HBox+ and the value of this
  option is passed to the \lstinline+HBox+ constructor\\
  edge & string & "ncline" & name of a connection command (se \MO/
  documentation)
  \\
  Dalign & string & "top" & vertical alignment of subtrees for trees that go
  down (the root on the top); the other possible
  values are \lstinline+"center"+ and \lstinline+"bot"+\\
  \bottomrule
\end{tabular}
\caption{Table of \MO/ tree options.}\label{tab:options}
\end{table}

Here is an example of using some of these options.

\begin{ExempleMP}
input huffman

beginfig(0);
string charList[];
numeric frequency[];
charList[1]:="s_1"; frequency[1]:=2;
charList[2]:="s_2"; frequency[2]:=4;
charList[3]:="s_3"; frequency[3]:=2;
charList[4]:="s_4"; frequency[4]:=12;
charList[5]:="s_5"; frequency[5]:=8;

newBinHuffman.myHuff(5)(charList,frequency)
"treemode(R)","treeflip(true)","hsep(1.5cm)", "edge(nccurve)" , "angleA(0)", "angleB(0)";
myHuff.c=origin;
drawObj(myHuff);
endfig;
\end{ExempleMP}
  
\section{Access to Nodes and Leaves}

To access the nodes and the trees, you can use the \lstinline+treeroot+ command
from \MO/, see the documentation for details.

\commande|ntreepos(Obj(«name»))(«int»,«int»,etc.)|\smallskip\index{ntreepos@\lstinline+ntreepos+}

The sequence of \meta{int} gives the choice of branch where the children are
numbered from 1 to $n$.  

The following example shows a use of this mecanism.
\begin{ExempleMP}
input huffman;
beginfig(0);
string charList[];
numeric frequency[];
charList[1]:="s_1"; frequency[1]:=2;
charList[2]:="s_2"; frequency[2]:=4;
charList[3]:="s_3"; frequency[3]:=2;
charList[4]:="s_4"; frequency[4]:=12;
charList[5]:="s_5"; frequency[5]:=8;

newBinHuffman.myHuff(5)(charList,frequency);
myHuff.c=origin;
ncarcbox(ntreepos(Obj(myHuff))(2,1,2))(ntreepos(Obj(myHuff))(2,2))
"linestyle(dashed evenly)", "nodesepA(5mm)", "nodesepB(5mm)" ;
drawObj(myHuff);
endfig;
\end{ExempleMP}

Of course, this only can be used in two steps, first build the tree, then
annotate it. 

You can also access the leaves and the nodes using names. During construction of
the tree, names are given to leaves and nodes. Because you may want to build
several Huffman trees, the trees are numbered. You can get the current tree
number with the following command:

\commande|get_huffmanTreeNbr()|\smallskip\index{get_huffmanTreeNbr@\lstinline+get_huffmanTreeNbr+}

The leaves are numbered during the construction, and the corresponding \MO/
object is named as follows:

\commande|leaf«leaf number»_«tree number»|\smallskip

The nodes are numbered during the construction, and the corresponding \MO/
object is named as follows:

\commande|node«node number»_«tree number»|\smallskip

Thanks to \MO/ you can annotate the tree using all the tools \MO/ provides. 

\begin{ExempleMP}
input huffman;
beginfig(0);
string charList[];
numeric frequency[];
charList[1]:="s_1"; frequency[1]:=2;
charList[2]:="s_2"; frequency[2]:=4;
charList[3]:="s_3"; frequency[3]:=2;
charList[4]:="s_4"; frequency[4]:=12;
charList[5]:="s_5"; frequency[5]:=8;

newBinHuffman.myHuff(5)(charList,frequency);
myHuff.c=origin;
ncarcbox(node2_1)(leaf4_1)
"linestyle(dashed evenly)", "nodesepA(5mm)", "nodesepB(5mm)" ;
drawObj(myHuff);
endfig;
\end{ExempleMP}


  





\section{Constructors}

The Huffman algorithm use only three constructors that you can redefine to adapt
the tree to your needs. Here are the three constructors (roughly commented in
French) defined in this package. We will no discus the code here but
you are free to redefine and adapt it.

\begin{mpcode}[title={Leaf Code}]
% style d’une feuille caractère et proba
vardef newHuffmanLeaf@#(expr ch)(expr v) text options=
    % @# est l’identifiant de la feuille
    % c est le caractère considéré (ou la chaine)
    % v est la proba ou l’entier associé
    save _text_v,
    _text_token,_height_v,_height_token,_height_max,_width_token,_width_v;
    picture _text_v,_text_token;
    % on calcule le height max des deux écritures pour faire deux boites de même
    % hauteur 
    _text_v := textext(v);
    _text_token := textext("$"&ch&"$");
    _height_v := abs((ulcorner _text_v) - (llcorner _text_v));
    _width_v := abs(urcorner _text_v - ulcorner _text_v);
    _height_token := abs(ulcorner _text_token - llcorner _text_token);
    _width_token := abs(urcorner _text_token - ulcorner _text_token);
    _height_max := max(_height_token,_height_v);
    % on fabrique deux boîtes vides aux bonne dimensions
    % et on ajoute un label au centre de celles-ci 
    if(show_leaf_values):    
        newEmptyBox.scantokens(str @# & "ch1")(_width_token+4,2_height_max)
        "framed(true)","fillcolor(_huffmanLeaf)", "filled(true)", options;
        ObjLabel.Obj(scantokens(str @# & "ch1"))(textext("$" & ch & "$"));
        newEmptyBox.scantokens(str @# & "ch2")(_width_v+4,2_height_max)
        "framed(true)", options;
        ObjLabel.Obj(scantokens(str @# & "ch2"))(textext(v));
        % on fixe relativement les coordonnées des deux boites pour qu’elles se % touchent 
        scantokens(str @# & "ch1").e=scantokens(str @# & "ch2").w;
        % on fabrique un container qui les regroupes et qui sera la feuille
        newContainer.@#(scantokens(str @# & "ch1"),scantokens(str @# & "ch2"));
    else:
        % si seulement le symbole
        newBox.@#(textext("$" & ch & "$"))
        "framed(true)","fillcolor(_huffmanLeaf)", "filled(true)", options;
    fi
enddef;
\end{mpcode}


\begin{mpcode}[title={Node Code}]
% style d’un nœud interne (non feuille) de l’arbre 
vardef newHuffmanNode@#(expr v) text options=
    newCircle.@#("") "circmargin(_node_size)",options;
    if(show_node_values):
    ObjLabel.Obj(scantokens(str @#))(textext(v));
    fi
enddef;
\end{mpcode}

\begin{mpcode}[title={Tree Code}]
% style de l’arbre binaire de Huffman
vardef newHuffmanBinTree@#(suffix theroot)(text subtrees) text options=
    % un simple arbre
    newTree.@#(theroot)(subtrees) "Dalign(top)" , "hbsep(0.3cm)",options;
    if(show_bits):
        % et on met 0 et 1 sur ses deux connections
        ObjLabel.Obj(@#)(btex 0 etex)
        "labpathid(1)", "laberase(true)", "labcolor(_huffmanBit)";
        ObjLabel.Obj(@#)(btex 1 etex)
        "labpathid(2)", "laberase(true)", "labcolor(_huffmanBit)";
    fi
enddef;
\end{mpcode}

\printbibliography
\printindex
\end{document}
