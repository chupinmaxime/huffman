
% LTeX: language=en

% huffman : draw binary huffman trees with MetaPost/MetaObj
%
% Originally written by Maxime Chupin <notezik@gmail.com>,
% 2023.
%
% Distributed under the terms of the GNU free documentation licence:
%   http://www.gnu.org/licenses/fdl.html
% without any invariant section or cover text.

\documentclass[english]{ltxdoc}

\input{huffman-preamble}

\usepackage[english]{babel}

\makeindex[title=Command Index, columns=2]



%\lstset{moredelim=*[s][\color{red}\rmfamily\itshape]{<}{>}}
%\lstset{moredelim=*[s][\color{blue}\rmfamily\itshape]{<<}{>>}}

\begin{document}

\title{{Huffman} : drawing binary Huffman trees with \hologo{METAPOST} and \MO/}
\author{Maxime Chupin, \url{notezik@gmail.com}}
\date{\today}

%% === Page de garde ===================================================
\thispagestyle{empty}
\begin{tikzpicture}[remember picture, overlay]
  \node[below right, shift={(-4pt,4pt)}] at (current page.north west) {%
    \includegraphics{fond.pdf}%
  };
\end{tikzpicture}%

\noindent
{\Huge \texttt{huffman}}\par\bigskip
\noindent
{\Large  drawing binary Huffman trees \\[0.2cm]with \hologo{METAPOST} and \MO/}\\[1cm]
\parbox{0.6\textwidth}{
  \begin{mplibcode}
    input huffman

    string charList[];
numeric frequency[];
show_bits:=false;
beginfig(1);
charList[4]:="a"; frequency[4]:=4;
charList[5]:="b"; frequency[5]:=5;
charList[3]:="c"; frequency[3]:=6;
charList[1]:="d"; frequency[1]:=7;
charList[2]:="e"; frequency[2]:=10;
charList[6]:="f"; frequency[6]:=10;
charList[7]:="g"; frequency[7]:=18;
charList[8]:="h"; frequency[8]:=40;
newBinHuffman.myHuff(8)(charList,frequency) "treemode(T)", "vbsep(0.3cm)";
myHuff.c=origin;
drawObj(myHuff);
endfig;
  \end{mplibcode}
}\hfill
\parbox{0.5\textwidth}{\Large\raggedleft
  \textbf{Contributor}\\
  Maxime \textsc{Chupin}\\
  \url{notezik@gmail.com}
}
\vfill
\begin{center}
  Version 0.1, 2023, April, 21th \\
  \url{https://plmlab.math.cnrs.fr/mchupin/huffman}
\end{center}
%% == Page de garde ====================================================
\newpage

%\maketitle

\begin{abstract}
  This \hologo{METAPOST} package allows to draw binary Huffman trees from two
  arrays : a string one, and a value one. It is based on \MO/ package which
  provides many tools to build trees in general.
\end{abstract}


\begin{center}
  \url{https://plmlab.math.cnrs.fr/mchupin/huffman}
  \url{https://github.com/chupinmaxime/huffman}
\end{center}

\tableofcontents

\bigskip

\begin{tcolorbox}[ arc=0pt,outer arc=0pt,
  colback=darkred!3,
  colframe=darkred,
  breakable,
  boxsep=0pt,left=5pt,right=5pt,top=5pt,bottom=5pt, bottomtitle =
  3pt, toptitle=3pt,
  boxrule=0pt,bottomrule=0.5pt,toprule=0.5pt, toprule at break =
  0pt, bottomrule at break = 0pt,]
  \itshape
  This package is in beta version---do not hesitate to report bugs, as well as requests for improvement.
\end{tcolorbox}

\section{Installation}

\huffman is on \ctan{} and can also be installed via the package manager of your
distribution.

\begin{center}
  \url{https://www.ctan.org/pkg/huffman}
\end{center}


\subsection{With \TeX live under Linux or macOS}

To install \huffman with \TeX Live, you will have to create the directory
\lstinline+texmf+ directory in your \lstinline+home+. 

\begin{commandshell}
mkdir ~/texmf
\end{commandshell}

Then, you will have to place the \lstinline+huffman.mp+ file in the 
\begin{center}
  \lstinline+~/texmf/metapost/huffman/+
\end{center}


Once this is done, \huffman will be loaded with the classic \MP
input code
\begin{mpcode}
input huffman
\end{mpcode}

\subsection{With Mik\TeX{} and Windows}

These two systems are unknown to the author of \huffman, so we
refer you to the Mik\TeX documentation concerning the addition of local packages:
\begin{center}
  \url{http://docs.miktex.org/manual/localadditions.html}
\end{center}



\subsection{Dependencies}


\huffman depends, of course on \MP~\cite{ctan-metapost}, as well as the packages \package{metaobj}~\cite{ctan-metaobj}
and---if \huffman is not used with \hologo{LuaLaTeX} and the \package{luamplib}
package---the \package{latexmp} package.

\section{Main Command}

The package \huffman provides one principal command (which is a \MO/ like
constructor):

\commande|newBinHuffmanTree.«name»(«sizeofarrays»)(«symbarray»,«valuearray»)|\smallskip\index{newBinHuffman@\lstinline+newBinHuffmanTree+}


\begin{description}
  \item[\meta{name}:] is the name of the object;
  \item[\meta{sizeofarray}:] is the size (integer) of the arrays;
  \item[\meta{symbarray}:] is the array of \lstinline+string+ containing the
  symboles;
  \item[\meta{valuearray}:] is the array of \lstinline+numeric+ containing the
  values associated to the symboles.
\end{description}



\begin{ExempleMP}
input huffman

beginfig(0);
string charList[];
numeric frequency[];
charList[1]:="a"; frequency[1]:=0.04;
charList[2]:="b"; frequency[2]:=0.05;
charList[3]:="c"; frequency[3]:=0.06;
charList[4]:="d"; frequency[4]:=0.07;
charList[5]:="e"; frequency[5]:=0.1;
charList[6]:="f"; frequency[6]:=0.1;
charList[7]:="g"; frequency[7]:=0.18;
charList[8]:="h"; frequency[8]:=0.4;

newBinHuffman.myHuff(8)(charList,frequency);
myHuff.c=origin;
drawObj(myHuff);
endfig;
\end{ExempleMP}

\printbibliography
\printindex
\end{document}
